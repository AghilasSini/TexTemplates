% Chapter Template

\chapter{State of the Art} % Main chapter title

\label{Chapter01} % Change X to a consecutive number; for referencing this chapter elsewhere, use \ref{ChapterX}

%----------------------------------------------------------------------------------------
%	SECTION 1
%----------------------------------------------------------------------------------------

\section{Expressive Speech Synthesis}
In \cite{doukhan:tel-01161731}
\begin{itemize}
\item Une des propositions centrales de ce travail est de définir, uti-
liser et mesurer l’impact de structures linguistiques opérant au delà du niveau de la phrase, par
opposition aux approches opérant sur des phrases isolées de leur contexte.
\item  segmentation des contes en épisodes.
\item détection des citations directes.
\item actes de dialogue.
 \item modes de communication. 	
 \item Les relations entre les annotations linguistiques et les propriétés prosodiques observées dans le corpus de parole sont décrites et modélisées
 \item Mis à part les choix lexicaux, la transmission d’images, d’émotions et d’intuitions se fait principalement via un processus de communication non verbale, qui peut se décrire par des intonations, des postures et des mimiques
 
 \item générer des instructions prosodiques permettant d’améliorer l’expressivité de la parole synthétisée (livres audio) 
\item thématique
	\begin{itemize}
		\item la linguistique de corpus
		\item le traitement automatisé du langage \item le traitement du signal	
		\item la modélisation prosodique
		\item la synthèse de parole.
\end{itemize}
\end{itemize}
%-----------------------------------
%	SECTION 2
%-----------------------------------
\section{Audio Books}

Nunc posuere quam at lectus tristique eu ultrices augue venenatis. Vestibulum ante ipsum primis in faucibus orci luctus et ultrices posuere cubilia Curae; Aliquam erat volutpat. Vivamus sodales tortor eget quam adipiscing in vulputate ante ullamcorper. Sed eros ante, lacinia et sollicitudin et, aliquam sit amet augue. In hac habitasse platea dictumst.

%-----------------------------------
%	SECTION 3
%-----------------------------------

\section{SynpaFlex Project}
Morbi rutrum odio eget arcu adipiscing sodales. Aenean et purus a est pulvinar pellentesque. Cras in elit neque, quis varius elit. Phasellus fringilla, nibh eu tempus venenatis, dolor elit posuere quam, quis adipiscing urna leo nec orci. Sed nec nulla auctor odio aliquet consequat. Ut nec nulla in ante ullamcorper aliquam at sed dolor. Phasellus fermentum magna in augue gravida cursus. Cras sed pretium lorem. Pellentesque eget ornare odio. Proin accumsan, massa viverra cursus pharetra, ipsum nisi lobortis velit, a malesuada dolor lorem eu neque.

%----------------------------------------------------------------------------------------
%	SECTION 4
%----------------------------------------------------------------------------------------
\subsection{Phonestyle in french}
In \citep{AnneCatherine2010PhoneStyle} .....
\subsection{Emphatic Stresses}
In \citep{Brognaux2014SynthesizingSC}
\subsection{Sport Commentaries}
\citep{brognaux}
\subsection{roots tool}
\citep{chevelu:hal-00974628}
\subsection{Prosody Modeling Techniques}
In \citep{key:article} show the the weakness and advantage of various prosody modeling techniques, also there impact on the speech synthesis systems as well as on emotional speech synthesis systems.
\subsubsection{Neural Network based  for modeling prosody}
In \citep{RamuReddy:2016:PMS:2839529.2839900}, the authors show that using PCPA(positional, contextual, phonological and articulatory) as input features of feed forward neural nets it's possible to predict  prosody parameters such as Intensity, intonation and duration in the syllable level.


\subsubsection{Prosody in Speech Synthesis}

In \citep{hirose2015speech}, they present different of view, for prosody modeling for getting a High Speech Quality.

In \citep{Rao:2012:PPT:2222511}

\subsection{data annotation}
In \citep{boeffard}

\begin{itemize}
\item Définir des informations linguistiques utiles pour améliorer la synthèse
\item Concevoir des systèmes permettant d’extraire automatiquement ces informations linguistiques à partir du texte
\item Mettre au point des règles de mise en correspondance des informations linguistiques en instructions prosodiques
\end{itemize}