%%%%%%%%%%%%%%%%%%%%%%%%%%%%%
% Auteur : Aghilas Sini %
% Cr�� le : 18/06/2013    %
% Version : 1.8             %
%%%%%%%%%%%%%%%%%%%%%%%%%%%%%%%%%
% Commandes pour la compilation	%
% latex cv.tex                 	%
% dvips -Ppdf -t a4 cv.dvi     	%
% ps2pdf cv.ps                 	%
%%%%%%%%%%%%%%%%%%%%%%%%%%%%%%%%%

\documentclass[a4paper,oneside]{resume}


\usepackage[english]{babel} 
\usepackage[latin1]{inputenc}  % les accents dans le fichier.tex

\usepackage{vmargin} %left, top, right, bottom
\usepackage{graphicx} % Pour ins�rer des images
\usepackage{wrapfig} % Pour placer les images

\setmarginsrb{2.2cm}{2.0cm}{2.0cm}{2.2cm}{0.96cm}{0.96cm}{0.96cm}{0.96cm}

\usepackage{relsize}

%%%%%%%%%%%%%%%%%%%%%%%%%%%%%%%%%%%%
% D�finition de quelques macros    %
%%%%%%%%%%%%%%%%%%%%%%%%%%%%%%%%%%%%

% ligne horizontale sur toute la page. Usage : \ligne{Largeur}
\newcommand{\ligne}[1]{\rule[0.5ex]{\textwidth}{#1}\\}
\newcommand{\interRubrique}{\smallskip}
\newcommand{\styleRub}[1]{\noindent\textbf{\large #1}\par}
\newcommand{\indentStd}{\noindent\hspace{\lenA}}



%%%%%%%%%%%%%%%%%%%%%%%%%%%%%%%%%%%
% Commandes Personnalis�es    	  %
%%%%%%%%%%%%%%%%%%%%%%%%%%%%%%%%%%%

% Personnalisation du titre avec un cadre gris�.
\newcommand{\mytitle}[1]{
    \begin{center}
    {\large \parashade[.87]{sharpcorners}{\textbf{#1 \vphantom{p\^{E}}}}}
    \end{center}
}



%%%%%%%%%%%%%%%%%%%%%%%%%%%%%%%%%%%%%
% L'environnement "rubrique" 
%
% Usage : \begin{rubrique}[Ind3.4entation]{Titre} [...] \end{rubrique}
% Ensuite, la premi�re colonne contient par exemple les dates, la seconde
% le descriptif.
% Par exemple :
%
% \begin{rubrique}{3.5cm}{pipotage}
% 1999--2000 	& ligne 1\\
% 		& ligne 2\\
% 1998--1999	& ligne 1\\
% [etc...]
% \end{rubrique}
%%%%%%%%%%%%%%%%%%%%%%%%%%%%%%%%%%%%%

\newenvironment{rubrique}[2][\linewidth] {
    \styleRub{#2}
    \setlength{\lenB}{#1}
    \setlength{\lenC}{\linewidth}
    \addtolength{\lenC}{-\lenA}
    \addtolength{\lenC}{-\lenB}
    \addtolength{\lenC}{-\parindent}
    \addtolength{\lenC}{-9pt}
    \indentStd\begin{tabular}[t]{p{\lenB}p{\lenC}}
}
{\end{tabular}}



%%%%%%%%%%%%%%%%%%%%%%%%%%%%%%%%%%%%%%%%%%%%
% Commandes utilisables dans le descriptif %
%					   %
% Modifiables � loisir... 		   %
%%%%%%%%%%%%%%%%%%%%%%%%%%%%%%%%%%%%%%%%%%%%
\newcommand{\projectname}[1]{\small{\textsl{#1}\ }}
\newcommand{\lieu}[1]{\small{\textsl{#1}\ }}
\newcommand{\activite}[1]{\textbf{#1}\ }
\newcommand{\comment}[1]{\textsl{#1}\ }
\newcommand*\bull{\ \ \raisebox{-0.365em}[-0.15em][-0.15em]{\textscale{4}{$\cdot$}} \ } % Custom bullet point for separating content


%%%%%%%%%%%%%%%%%%%%%%%%%%%%%%%%%%%%%%%%%%
% D�but du CV proprement dit (ouf ! :) ) %
%%%%%%%%%%%%%%%%%%%%%%%%%%%%%%%%%%%%%%%%%%
\name{Aghilas SINI} % Your name
\address{(+33) 07~$\cdot$~77~$\cdot$~33~$\cdot$~37~$\cdot$~64 \\ sini.aghilas@gmail.com} % Your address
\address{615 Rue du Jardin Botanique \\ Villers les Nancy , 54600 France} % Your secondary addess (optional)
\address{\textbf{\large{Research And Development Engineer	(2 ans d'exp�rience)}}} % Your phone number and email
%\address{(+33) 03~$\cdot$~83~$\cdot$~59~$\cdot$~20~$\cdot$~97 \\ aghilas.sini@inria.fr} % Your phone number and email

\pagestyle{empty} % pour ne pas indiquer de num�ro de page...
\begin{document}
\newlength{\lenA} % indentation au d�but d'une ligne
\setlength{\lenA}{0.cm}
\newlength{\lenB} % Taille champ dates
\newlength{\lenC} % Taille champ description



%%%%%%%%%%%%%%%%%%%%%%%%%%%%%%%%%
% en-t�te 			%
%%%%%%%%%%%%%%%%%%%%%%%%%%%%%%%%%	

   
 
\begin{rSection}{Skills}
\end{rSection}

\begin{rubrique}[0.5cm]{D�velopement web Front}
	& JavaScript, jQuery, Ajax\\
	& CSS3, Boostrap 3, HTML5
\end{rubrique}
\interRubrique
\\
\begin{rubrique}[0.5cm]{Simulation/Calcule Num�rique}
	& Matlab, Octave \\
\end{rubrique}
\interRubrique
\\
\begin{rubrique}[0.5cm]{G�stion de G�n�ration/ Scafolding}
	& Bower, Yeoman,Grunt, NPM\\
\end{rubrique}
\interRubrique
\\
\begin{rubrique}[0.5cm]{Machine Learning/Data Science/Robotic}
	& DL4J, Theano, TensorFlow, R, ROS\\
\end{rubrique}
\interRubrique
\\
\begin{rubrique}[0.5cm]{Programming}
	&  Shell script, Windows Batch script, Jython\\
	& C, C++, Java, Python\\
\end{rubrique}
\interRubrique
\\
\begin{rubrique}[0.5cm]{Protocols \& APIs}
	&  XML, JSON, SOAP, REST\\	
\end{rubrique}
\interRubrique
\\
\begin{rubrique}[0.5cm]{M�thode / Analyse}
	&  M�thode UML, Agile(SCRUM)	
\end{rubrique}
\interRubrique
\\
\begin{rubrique}[0.5cm]{Gestion de configuration/Industrialisation}
	& Git, Maven3, Jenkins\	
\end{rubrique}
\interRubrique
\\



\begin{rSection}{Education}
\end{rSection} 
 \begin{rubrique}[2cm]{}
    2014
    & \activite{Master 2 \-- Artificial Intelligence, Pattern Recognition and Robotics}\\
    & \lieu{University Paul SABATIER- TOULOUSE.} \comment{France} \\ \\
   
    2013
    & \activite{Master 1 \-- Real Time Systems Engineering} \\
     & \lieu{University Paul SABATIER- TOULOUSE.}\comment{France}  \\ \\
  
    2011 		
     & \activite{B.Sc \--	Control System and Automation}\\
    & \lieu{University Mouloud MAMMERI- TIZI OUZOU.} \comment{Algeria}
\end{rubrique}

\newpage

\begin{rSection}{Experience}
\end{rSection}
\begin{rubrique}[21cm]{\Large{January 2016 - March 2016}}
\activite{Enseignant Vacataire}\\
\lieu{IUT Charlemagne, Universit� de Lorraine, Nancy.} \comment{France}\\
\end{rubrique}
\interRubrique
\\
\begin{rubrique}[0.4cm]{\,\underline{\small{Principaux domaines d'intervantion}}}

& \bull Introduction � la programmation Web \-- Front end \\
& \bull Cours avanc� en JavaScript, JQuery, Ajax, Rest.\\
& \bull Mini Projet. 
\end{rubrique}

\interRubrique
\begin{rubrique}[0.4cm]{\,\underline{\small{Environnement Technique}}}
	&\bull JavaScript, JQuery, Ajax, JSON, Rest. 
\end{rubrique}
\interRubrique
\\
%------------------------------------------------



%------------------------------------------------

\begin{rubrique}[21cm]{\Large{January 2016 - Present}} \activite{Ing�nieur de Recherche et D�vellopement}\\ \lieu{LORIA\--INRIA Nancy Laboratory.} \comment{France}\\ \projectname{IFCASL Project Individualized Feedback for Computer-Assisted Spoken Language 
  Learning}
\end{rubrique}
\interRubrique

\begin{rubrique}[0.4cm]{\,\underline{\small{Principaux domaines d'intervantion}}}
		 &\bull Modification and re-synthesis  of learner audio samples using teacher audio samples based on Pitch Synchronous Overlap and Add algorithm\\
		&\bull Feedback to
correct devoicing of final consonants in French  spoken by German learners\\
		&\bull \,Improving speech text alignment for language learning using deep neural network, training network with theano and decoding with DL4J (IFCASL Corpus)\\ 		
		&\bull Extraction of speech features  for pitch detection using deep neural network
\end{rubrique}
\interRubrique
\\
\begin{rubrique}[0.4cm]{\,\underline{\small{Environnement Technique}}}
	&\bull Java, Python, TensorFlow, Theano, R
\end{rubrique}
\interRubrique
\\
\begin{rubrique}[21cm]{\Large{November 2014 - December 2015}}
\activite{Ing�nieur de Recherche et D�vellopement}\\
\lieu{LORIA\--INRIA Nancy Laboratory.} \comment{France}\\
\projectname{ORTOLANG Project Open Resources and TOols for LANGuage}
\end{rubrique}
\interRubrique
\\
\begin{rubrique}[0.4cm]{\,\underline{\small{Principaux domaines d'intervantion}}}
&\bull Development of syntactic-semantic analyser for spoken documents in  French language \-- open source tool J\--Safran released by LORIA. Includes dependency parser for oral speech and inclusion of inference of partial semantic role labels on top of syntactic parsing\\
	&\bull Tool for semi\--automatic alignment of speech and textual corpus \-- open source tool JTRANS relesead by LORIA. Working on correction of speech text alignment
 around silence segments.\\
	&\bull \, Interactive tool for speech signal processing and phonetics \-- open source tool JSnoori released by LORIA. Development of module for pitch estimation\\
\end{rubrique}
\interRubrique
\\
\begin{rubrique}[0.4cm]{\,\underline{\small{Environnement Technique}}}
& \bull Java, Jenkins, Git, Linux, Yoeman, Grunt, Bower ,NPM  

\end{rubrique}
\end{document}

