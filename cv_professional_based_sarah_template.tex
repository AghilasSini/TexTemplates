%%%%%%%%%%%%%%%%%%%%%%%%%%%%%
% Auteur : Aghilas Sini %
% Cr�� le : 18/06/2013    %
% Version : 1.8             %
%%%%%%%%%%%%%%%%%%%%%%%%%%%%%%%%%
% Commandes pour la compilation	%
% latex cv.tex                 	%
% dvips -Ppdf -t a4 cv.dvi     	%
% ps2pdf cv.ps                 	%
%%%%%%%%%%%%%%%%%%%%%%%%%%%%%%%%%

\documentclass[a4paper,oneside]{resume}


\usepackage[english]{babel} 
\usepackage[latin1]{inputenc}  % les accents dans le fichier.tex

\usepackage{vmargin} %left, top, right, bottom
\usepackage{graphicx} % Pour ins�rer des images
\usepackage{wrapfig} % Pour placer les images

\setmarginsrb{2.2cm}{0.0cm}{2.2cm}{2.2cm}{0.006cm}{0.96cm}{0.96cm}{0.96cm}

\usepackage{relsize}

%%%%%%%%%%%%%%%%%%%%%%%%%%%%%%%%%%%%
% D�finition de quelques macros    %
%%%%%%%%%%%%%%%%%%%%%%%%%%%%%%%%%%%%

% ligne horizontale sur toute la page. Usage : \ligne{Largeur}
\newcommand{\ligne}[1]{\rule[0.5ex]{\textwidth}{#1}\\}
\newcommand{\interRubrique}{\smallskip}
\newcommand{\styleRub}[1]{\noindent\textbf{\large #1}\par}
\newcommand{\indentStd}{\noindent\hspace{\lenA}}



%%%%%%%%%%%%%%%%%%%%%%%%%%%%%%%%%%%
% Commandes Personnalis�es    	  %
%%%%%%%%%%%%%%%%%%%%%%%%%%%%%%%%%%%

% Personnalisation du titre avec un cadre gris�.
\newcommand{\mytitle}[1]{
    \begin{center}
    {\large \parashade[.87]{sharpcorners}{\textbf{#1 \vphantom{p\^{E}}}}}
    \end{center}
}



%%%%%%%%%%%%%%%%%%%%%%%%%%%%%%%%%%%%%
% L'environnement "rubrique" 
%
% Usage : \begin{rubrique}[Ind3.4entation]{Titre} [...] \end{rubrique}
% Ensuite, la premi�re colonne contient par exemple les dates, la seconde
% le descriptif.
% Par exemple :
%
% \begin{rubrique}{3.5cm}{pipotage}
% 1999--2000 	& ligne 1\\
% 		& ligne 2\\
% 1998--1999	& ligne 1\\
% [etc...]
% \end{rubrique}
%%%%%%%%%%%%%%%%%%%%%%%%%%%%%%%%%%%%%

\newenvironment{rubrique}[2][\linewidth] {
    \styleRub{#2}
    \setlength{\lenB}{#1}
    \setlength{\lenC}{\linewidth}
    \addtolength{\lenC}{-\lenA}
    \addtolength{\lenC}{-\lenB}
    \addtolength{\lenC}{-\parindent}
    \addtolength{\lenC}{-9pt}
    \indentStd\begin{tabular}[t]{p{\lenB}p{\lenC}}
}
{\end{tabular}}



%%%%%%%%%%%%%%%%%%%%%%%%%%%%%%%%%%%%%%%%%%%%
% Commandes utilisables dans le descriptif %
%					   %
% Modifiables � loisir... 		   %
%%%%%%%%%%%%%%%%%%%%%%%%%%%%%%%%%%%%%%%%%%%%
\newcommand{\projectname}[1]{\small{\textsl{#1}\ }}
\newcommand{\lieu}[1]{\small{\textsl{#1}\ }}
\newcommand{\activite}[1]{\textbf{#1}\ }
\newcommand{\comment}[1]{\textsl{#1}\ }
\newcommand*\bull{\ \ \raisebox{-0.365em}[-0.15em][-0.15em]{\textscale{4}{$\cdot$}} \ } % Custom bullet point for separating content


%%%%%%%%%%%%%%%%%%%%%%%%%%%%%%%%%%%%%%%%%%
% D�but du CV proprement dit (ouf ! :) ) %
%%%%%%%%%%%%%%%%%%%%%%%%%%%%%%%%%%%%%%%%%%
\name{Aghilas SINI} % Your name
\address{(+33) 07~$\cdot$~77~$\cdot$~33~$\cdot$~37~$\cdot$~64 \\ sini.aghilas@gmail.com} % Your address
\address{615 Rue du Jardin Botanique \\ Villers les Nancy , 54600 France} % Your secondary addess (optional)
\address{\textbf{\large{Research And Development Engineer	(2 ans d'exp�rience)}}} % Your phone number and email
%\address{(+33) 03~$\cdot$~83~$\cdot$~59~$\cdot$~20~$\cdot$~97 \\ aghilas.sini@inria.fr} % Your phone number and email

\pagestyle{empty} % pour ne pas indiquer de num�ro de page...
\begin{document}
\newlength{\lenA} % indentation au d�but d'une ligne
\setlength{\lenA}{0.cm}
\newlength{\lenB} % Taille champ dates
\newlength{\lenC} % Taille champ description



%%%%%%%%%%%%%%%%%%%%%%%%%%%%%%%%%
% en-t�te 			%
%%%%%%%%%%%%%%%%%%%%%%%%%%%%%%%%%	

   
 
\begin{rSection}{Skills}
\end{rSection}
\begin{rubrique}[0.5cm]{D�velopement web Front}
	&\bull JavaScript, jQuery, Ajax\\
	&\bull CSS3, Boostrap 3, HTML5
\end{rubrique}
\interRubrique
\\ 

\begin{rubrique}[0.5cm]{Simulation/Calcule Num�rique}
	&\bull Matlab, Octave \\
\end{rubrique}
\interRubrique
\\ 

\begin{rubrique}[0.5cm]{G�stion de G�n�ration/ Scafolding}
	&\bull Bower, Yeoman,Grunt, NPM\\
\end{rubrique}
\interRubrique
\\ 

\begin{rubrique}[0.5cm]{Machine Learning/Data Science/Robotic}
	&\bull DL4J, Theano, TensorFlow, R, ROS\\
\end{rubrique}
\interRubrique
\\ 

\begin{rubrique}[0.5cm]{Programming}
	& \bull Shell script, Windows Batch script, Jython\\
	& \bull C, C++, Java, Python\\
\end{rubrique}
\interRubrique
\\
 
\begin{rubrique}[0.5cm]{Protocols \& APIs}
	&  \bull XML, JSON, SOAP, REST\\	
\end{rubrique}
\interRubrique
\\
 
\begin{rubrique}[0.5cm]{M�thode / Analyse}
	&  \bull M�thode UML, Agile(SCRUM)	
\end{rubrique}
\interRubrique
\\ 

\begin{rubrique}[0.5cm]{Gestion de configuration/Industrialisation}
	& Git, Maven3, Jenkins, Cobertura\	
\end{rubrique}
\interRubrique
\\

\begin{rSection}{Education}
\end{rSection} 
 \begin{rubrique}[2cm]{}
    2014
    & \activite{Master 2 \-- Artificial Intelligence, Pattern Recognition and Robotics}\\
    & \lieu{University Paul SABATIER- TOULOUSE.} \comment{France} \\ \\ 
    2013
    & \activite{Master 1 \-- Real Time Systems Engineering} \\
     & \lieu{University Paul SABATIER- TOULOUSE.}\comment{France}  \\ \\ 
    2011 		
     & \activite{B.Sc \--	Control System and Automation}\\
    & \lieu{University Mouloud MAMMERI- TIZI OUZOU.} \comment{Algeria}\\ \\ 
\end{rubrique}
%\interRubrique
%\\
\newpage
\begin{rSection}{Experience}
\end{rSection}
\begin{rubrique}[21cm]{\Large{January 2016 - March 2016}}
\activite{Enseignant Vacataire}\\
\lieu{IUT Charlemagne, Universit� de Lorraine, Nancy.} \comment{France}\\
\end{rubrique}
\interRubrique
\\

\begin{rubrique}[0.4cm]{\,\underline{\small{Principaux domaines d'intervantion}}}
& \bull Introduction � la programmation Web \-- Front end \\
& \bull Cours avanc� en JavaScript, JQuery, Ajax, Rest.\\
& \bull Mini Projet. 
\end{rubrique}
\interRubrique
\\

\begin{rubrique}[0.4cm]{\,\underline{\small{Environnement Technique}}}
	&\bull JavaScript, JQuery, Ajax, JSON, Rest. 
\end{rubrique}
\interRubrique
\\

\begin{rubrique}[21cm]{\Large{January 2016 - Present}} \activite{Ing�nieur de Recherche et D�vellopement}\\ \lieu{LORIA\--INRIA Nancy Laboratory.} \comment{France}\\ \projectname{IFCASL Project Individualized Feedback for Computer-Assisted Spoken Language 
  Learning}
\end{rubrique}
\interRubrique
\\

\begin{rubrique}[0.4cm]{\,\underline{\small{Principaux domaines d'intervantion}}}
		 &\bull D�vellopement d'outils pour l'aide  � l'apprentissage de langue\\
		&\bull D�vellopement de scripts pour l'analyse de corpus de parole.\\
		&\bull \,Impl�mentation de module pour modification de signal de parole.\\ 		
		&\bull Impl�mentation de r�seau de neuron approfondi pour l'alignment texte parole.
\end{rubrique}
\interRubrique
\\

\begin{rubrique}[0.4cm]{\,\underline{\small{Environnement Technique}}}
	&\bull Java, Python \\
	&\bull TensorFlow, Theano\\
	&\bull  R, Jython, Windows Batch script,Shell script\\
\end{rubrique}
\interRubrique
\\

\begin{rubrique}[16cm]{\Large{November 2014 - December 2015}}
\activite{Ing�nieur de Recherche et D�vellopement}\\
\lieu{LORIA\--INRIA Nancy Laboratory.} \comment{France}\\
\projectname{ORTOLANG Project Open Resources and TOols for LANGuage}
\end{rubrique}
\interRubrique
\\

\begin{rubrique}[0.4cm]{\,\underline{\small{ Principaux domaines d'intervantion}}}
&\bull Maintenance �volutive d'outils de trgaitement et d'analyse du signal de parole.\\
&\bull Re\--factoring du code, de l'architecture.\\
	&\bull  D�vellopement de Portail web pour les outils.\\
	&\bull  Ecriture de test unitaire, covering, d'integration continue.\\
\end{rubrique}
\interRubrique
\\

\begin{rubrique}[0.4cm]{\,\underline{\small{Environnement Technique}}}
& \bull Shell Script, Jython, Java, Linux.\\
& \bull Jenkins (INRIA platform), Cobertura.\\  
& \bull Yeoman, Bower, NPM, Grunt, Git.\\
& \bull HTML5,CC3, JavaScript.\\
\end{rubrique}

\newpage
\begin{rubrique}[16cm]{\Large{ Mars 2014 -Ao�t 2014  }}
   \activite{Stage d'ing�nieur de recherche}\\
	\lieu{LORIA-INRIA Laboratory Nancy.}\\
\end{rubrique}
\interRubrique
\\

\begin{rubrique}[0.4cm]{\,\underline{\small{Principaux domaines d'intervantion}}}
	& \bull Control of a mobile robot movements to localize a sound source as quickly as possible. The belief about the source position is represented by a discrete grid and a dynamic programming algorithm  was introduced to find the optimal robot motion minimizing the entropy of the grid.\\
	\end{rubrique} 
\interRubrique
\\

\begin{rubrique}[0.4cm]{\,\underline{\small{Environnement Technique}}}
	& \bull C/C++, Python\\
	& \bull Robot Audition HARK, ROS\\
	& \bull Kinect, TurtleBot, Linux. \\
\end{rubrique}
\interRubrique
\\

\begin{rSection}{Publication}
\begin{rubrique}[0.4cm]{}
&\bull E. Vincent, A. Sini and F. Charpillet, "Audio source localization by optimal control of a mobile robot", Acoustics, Speech and Signal Processing (ICASSP), 2015 IEEE International Conference on, South Brisbane, QLD, 2015, pp. 5630-5634.\\
\end{rubrique} 
\interRubrique
\\
\end{rSection}

\begin{rSection}{Language}
\begin{rubrique}[16cm]{}
	\textbf{Kabyle} \, (native)\,
	\textbf{French} \, (fluent)\,  
    \textbf{English} \,(intermediate)\,
    \textbf{German}\,  (beginner)
\end{rubrique}        
\interRubrique
\\

\end{rSection}
\begin{rSection}{Referents}
\begin{rubrique}[3.4cm]{}
\textbf{Yves Laprie} & Research Director, CNRS Multispeech\\      
						& yves.laprie@loria.fr \\
\textbf{Denis Jouvet}  & Research Director, INRIA Multispeech\\
					& denis.jouvet@inria.fr \\
\end{rubrique} 
\interRubrique
\\
\end{rSection}



\end{document}

