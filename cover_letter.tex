\documentclass[10pt,a4paper]{letter}
\usepackage[utf8]{inputenc}
\usepackage[english]{babel}
\usepackage{amsmath}
\usepackage{amsfonts}
\usepackage{amssymb}
\usepackage{hyperref}
\begin{document}
Dear Axel Orgogozo and Laurence Deviller,\\

		I am writing to express my interest in the open PhD position titled " Deep Learning for emotion classification in speech ", posted at \href{http://www.anrt.asso.fr/com/imgAdmin/1459332134891.pdf}{association nationale recherche technologie} website. I am research engineer with the Multispeech project-team at LORIA laboratory Nancy, France. Pursuing my interest in speech signal processing I have been working with the Multispeech team for more than two years. I believe that this position will be a great opportunity to pursue a PhD in my area of interest. 
		
		At LORIA I have been working in area of Computer-Assisted Spoken Language Learning focusing on French spoken by German learners. my contribution is towards both the signal processing and machine learning aspects including speech feature extraction, speech modification and dynamic programming. Previously, I worked on speech text alignment which introduce me to methods in speech recognition. My master's intern-ship with Emmanuel Vincent and François Charptier, explored audio source localisation by mobile robot.A novel approach based on grid occupancy was proposed, this work got published in IEEE ICASSP 2015 Conference.  This experiences demonstrate my research skills and exposure. My MSc courses on signal processing, pattern recognition and machine learning laid the foundation for these.   

		
		While developing the JSnoori, JTrans and JSafran tools at LORIA, I gained expertise in different programming languages like Java, Python and JavaScript. At the same time I got a hands on experience in various algorithms and programming paradigms.And I got well acquainted 	to cluster computing environment	and tools for coding within team project. Recently I was introduced to the Theano and DL4J frameworks for implementation of deep learning architectures.Similarly during MSc intern-ship, I worked with Kinect based robot audio acquisition. These demonstrate my ability to easily adapt to new R\&D environments.    
		
		The above experiences have made me well aware of the challenges of research. As an individual I also got the opportunity to develop my interpersonal communication and team 		
skills.Working on multi-disciplinary projects involved coordination within  team as well as collaboration with other teams. Giving courses to students was one of the great experience that I had, which taught me to communicate to a large group. I am confident that this will enable me to serve me well in your project.

	Thank you for your time. Looking forward to discuss with you.
	
Sincerely,

Aghilas SIN 
	
		
\end{document}