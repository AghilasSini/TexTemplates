\documentclass[10pt,a4paper]{letter}
\usepackage[utf8]{inputenc}
\usepackage[english]{babel}
\usepackage{amsmath}
\usepackage{amsfonts}
\usepackage{amssymb}
\usepackage{hyperref}
\begin{document}

I am writing to express my interest in the open position of technology transfer engineer at Inria Lille, posted on the \href{http://www.inria.fr/institut/recrutement-metiers/offres/cdd/\%28view\%29/details.html?id=PUQFK026203F3VBQB6G68LO2G&ContractType=5033&SUBDEPT1=2&LG=FR&Resultsperpage=20&nPostingID=10527&nPostingTargetID=16850&option=52&sort=DESC&nDepartmentID=2}{Inria} website. I am currently a research engineer with the Multispeech team at LORIA-INRIA laboratory Nancy, France. Submitting this application with all respect I believe that my experience and my quest for knowledge qualifies me for this position.

At LORIA-INRIA I have been working on development of open source software for speech and language processing, including JSnoori, JTrans and JSafran. On one hand these projects address different research problems and demonstrate my analytical abilities and exposure to development of algorithms and models. And on the other hand they also showcase my strong background in different programming paradigms and languages (C/C++, Java, Python and JavaScript). At the same time these being team projects with deliverables I have an experience in development and collaboration over platforms like InriaForge (GForge), Github and CI Inria for continuous integration. Further being part of the process of update in requirements, architecture and user interface design I acquired skills in software development cycle. With the research side of the projects I got well acquainted to cluster computing environment and frameworks for machine learning and data analysis. 

My MSc internship with 
Emmanuel Vincent and Fran\c{c}ois Charptier 
gave me an opportunity to apply my knowledge in robotics to the real life problem of robot audition. This involved modelling of the problem of localisation of a sound source by a mobile robot and implementation of decision and control algorithms which enable the robot to move towards the sound source. After my internship I worked in other interesting areas of signal processing and machine learning such as  feature extraction, speech modification and re-synthesis and dynamic programming and alignment problems in speech recognition. These demonstrate my ability to easily adapt to new research and development environments.
 
The above experiences have made me well aware of the challenges of research. As an individual I also got the opportunity to develop my interpersonal communication and team skills.Working on multi-disciplinary projects involved coordination within team as well as collaboration with other teams. Giving courses to students was one of the great experience that I had, which taught me to communicate to a large group. I am confident that this will enable me to serve me well in your project.
 Thank you for your time. Looking forward to discuss with you.
 Sincerely,
Aghilas SIN
 \end{document}
