\documentclass{beamer}

% Choose how your presentation looks.
%
% For more themes, color themes and font themes, see:
% http://deic.uab.es/~iblanes/beamer_gallery/index_by_theme.html
%
\mode<presentation>
{
  \usetheme{Darmstadt}      % or try Darmstadt, Madrid, Warsaw, ...
  \usecolortheme{beaver} % or try albatross, beaver, crane, ...
  \usefonttheme{serif}  % or try serif, structurebold, ...
  \setbeamertemplate{navigation symbols}{}
  \setbeamertemplate{caption}[numbered]
} 

\usepackage[english]{babel}
\usepackage[utf8x]{inputenc}
\title[Background and Literature]{Data Annotation And Prosody Prediction In TTS}
\author{Aghilas Sini}
\institute{Université de Rennes 1}
\date{13/01/2017}

\setbeamertemplate{bibliography item}{}

%remove line breaks
\setbeamertemplate{bibliography entry title}{}
\setbeamertemplate{bibliography entry location}{}
\setbeamertemplate{bibliography entry note}{}



\begin{document}

%============================================================%
% Title Frame 
%============================================================%
\begin{frame}
  \titlepage
\end{frame}



\section{Problem}
%============================================================%
%
%============================================================%
\subsection{Overview}
\begin{frame}{Overview}

\end{frame}

%============================================================%
%
%============================================================%
\section{Data Annotation}
\begin{frame}{Data Annotation}


\end{frame}
%============================================================%
%
%============================================================%
\subsection{NEB Corpus}
\begin{frame}

\end{frame}
%============================================================%
%
%============================================================%
\section{Prosody Prediction}
\begin{frame}{Prosody Prediction}
	
\end{frame}
%============================================================%
%
%============================================================%
\begin{frame}

\end{frame}

\subsection{Key Problems In Text Processing and Speech Processing}

\begin{frame}{Key Problems In Text Processing}

\begin{itemize}
\item Neuro-Endotrainer tracking and evaluation
\item Neuro-endoscopy tool tracking(Aux camera and Endoscopy camera)
\item  Micro-suturing skills assessment - effectiveness(images) and Dexterity(video - activity detection and scoring)
\item Drilling Skill assessment effectiveness(images) and Dexterity(video - activity detection)
\end{itemize}
% Commands to include a figure:
%\begin{figure}
%\includegraphics[width=\textwidth]{your-figure's-file-name}
%\caption{\label{fig:your-figure}Caption goes here.}
%\end{figure}

\end{frame}

\begin{frame}{Key Problems In Speech Processing}
\end{frame}

\section{Related Word}
\subsection{Related Work - I}
\begin{frame}{Related Work and Literature review - I}
\Large The main focus so far has been only the Neuro-endotrainer
\begin{itemize}
\item It started with building background on Image Processing and Computer Vision:
\begin{itemize}
\item \large Notes on DIA - Basics of image representation, filtering operations, Image Warping
\item \large Mooc on Udacity - Math behind Canny Edge detection and Hough Transform
\item \large Another Mooc on CV - just specific topics from that made me comfortable with the math. - eg: SIFT descriptor etc
\item \large Notes on CV that led me to study the math behind
projective geometry - mainly Hartley and Zisserman
\item \large Getting used to coding in openCV
\end{itemize}
\end{itemize}
\end{frame}

\subsection{Related Work - II}
\subsection{Geared Towards Prosody Characterization}

\subsection{Future scope of work}

\begin{frame}{Suggested problems to work on}
\Large Currently the following problems are to be addressed
\begin{itemize}
\item Endotrainer related:
\begin{itemize}
\item \Large Identify tugging of the ring onto a peg.
\item \Large  Endoscopic Camera Evaluation - The tool is to be tracked and determined whether the tool is in the centre of the field or exiting the field and evaluate if it hits the peg.
\end{itemize}

\item Micro-suturing related (Image and Video)
\item Drill related (Image and Video)
\end{itemize}
\end{frame}
\begin{frame}
\centering \LARGE THANKYOU
\end{frame}

\subsection{References}
\begin{frame}{References}
\begin{itemize}
  \item Dill, K. A.; Truskett, T. M.; Vlachy, V.; Hribar-Lee, B. Modeling Water, The Hydrophobic Effect, \& Ion Solvation Annu. Rev. Biophys. Biomol. Struct. \textbf{2005}, 34, 179-199
  \item Silverstein, K. A. T.; Haymet, A. J. D.; Dill, K. A. The Strength of Hydrogen Bonds in Liquid Water and Around Nonpolar SOlutes J. Am. Chem. Soc. \textbf{2000}, 122, 8037-8041
  \item Silverstein, K. A. T.; Haymet, A. J. D.; Dill, K. A. A Simple Model of Water and the Hydrophobic Effect J. Am. Chem. Soc. \textbf{1998}, 120, 3166-3175
\end{itemize}
\end{frame}

\begin{frame}{References (Continued)}
\begin{itemize}
  \item Urbica, T.; Vlacy, V.; Kalyuzhnyi, Y. V.; Dill, K. A. An Imporoved Thermodynamic Peterbation Theory for Mercedes-Benz Water J. Chem. Phy. \textbf{2007}, 127, 1-4
  \item Silverstein, K. A. T.; Haymet, A. J. D.; Dill, K. A. Molecular Model of Hydrophobic Solvation J. Chem. Phys. \textbf{1999}, 111(17), 8000-8009 
\end{itemize}
\end{frame}



\section{Summary}

\subsection{Raised Questions?}
\begin{frame}{Questions?}
\begin{itemize}
  \item Study MB water to understand peculiar properties it possesses.
  \item Running various simulations to test MB water and Monte Carlo/Molecular Dynamics simulation methods.
  \item Testing results from previously published journals.
  \item Research can be used as a teaching resource and source of information for scientific community.
\end{itemize}
\end{frame}

\subsection{Work Plan}
\begin{frame}{Schedule For Next 6 Months?}
\center{\textbf{Planning the unplannable?}}


\url{https://www.officetimeline.com/gantt-chart-template/gantt-download}

\end{frame}



\end{document}
