\documentclass[10pt,a4paper]{letter}
\usepackage[utf8]{inputenc}
\usepackage[english]{babel}
\usepackage{amsmath}
\usepackage{amsfonts}
\usepackage{amssymb}
\usepackage{hyperref}
\begin{document}
Dear Axel Orgogozo and Laurence Deviller,\\

		I am writing to express my interest in the open PhD position titled "Deep Learning for emotion classification in speech", posted at \href{http://www.anrt.asso.fr/com/imgAdmin/1459332134891.pdf}{ANRT} website. I am a research engineer with the Multispeech project-team at LORIA laboratory Nancy, France. Pursuing my interest in speech signal processing I have been working with the Multispeech team for more than two years. I believe that this position will be a great opportunity to pursue a PhD in my area of interest. 
		
		At LORIA I have been working in the area of Computer-Assisted Spoken Language Learning focusing on French spoken by German learners. My contribution is towards both the signal processing and machine learning aspects including speech feature extraction, speech modification and dynamic programming. Previously, I worked on speech text alignment which was a good experience for me in speech recognition. My master's internship with Emmanuel Vincent and François Charpillet, explored audio source localisation by mobile robot. A novel approach based on grid occupancy was proposed. This work got published in IEEE ICASSP 2015 Conference. These experiences have enhanced my research skills. My MSc courses on signal processing, pattern recognition and machine learning laid the foundation for these.   

		
		While contributing to the development of JSnoori, JTrans and J\-Safran software at LORIA, as mentioned in my CV, I gained expertise in different programming languages like Java, Python and JavaScript. At the same time I got the opportunity to program various algorithms and paradigms. And I got well acquainted 	to cluster computing environment	and tools for collaborative coding within team projects. Recently, I was introduced to the Theano and DL4J frameworks for implementation of deep learning architectures while investigating combination of pitch estimation with  neural networks. Similarly during MSc internship, I worked with Kinect based robot audio acquisition. These demonstrate my ability to easily adapt to new R\&D environments.    
		
		The above experiences have made me well aware of the challenges of research. As an individual I also got the opportunity to develop my interpersonal communication and team 		
skills. Working on multi-disciplinary projects involved coordination within  team as well as collaboration with other teams. Giving courses to students was a great experience, which taught me to communicate to a large group. Participation in team seminars and project meetings gave me another degree of experience. I am confident that this will enable me to be efficient in your project.

	Being in the domain of research and working  with researchers for over two years gave me the urge to pursue a research career. The challenges and the perpetual evolution only increases my desire to continue in this domain. I believe that the PhD position will contribute enormously to my future research plan.
 
	Thank you for your time. Looking forward to discuss with you in this regard.
	
Sincerely,

Aghilas SINI
	
		
\end{document}