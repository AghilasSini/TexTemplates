\documentclass[10pt,a4paper]{letter}
\usepackage[utf8]{inputenc}
\usepackage[english]{babel}
\usepackage{amsmath}
\usepackage{amsfonts}
\usepackage{amssymb}
\usepackage{hyperref}
\begin{document}
Dear Philip GARNER,\\


		I am writing to express my interest in the open PhD position titled "Multilingual Affective Speech Synthesis ", posted at \href{https://www.idiap.ch/education-and-jobs}{Idiap Research Institute} website. I am currently a research engineer with the Multispeech team at LORIA-INRIA laboratory Nancy, France, where I have been pursuing my interest in speech signal processing for the past two years. I respectfully submit this letter of application for the PhD position at 
Idiap since I believe my experience so far and my quest for knowledge 
make me qualified to meet the needs of the program. 
		
		At LORIA\--INRIA I have been working in the area of Computer-Assisted Spoken Language Learning to assist German speakers to learn French. My contribution is towards both the signal processing and machine learning aspects including speech feature extraction, speech modification and dynamic programming. Previously, I worked on speech text alignment which was a good experience for me in speech recognition. My master's internship with Emmanuel Vincent and François Charpillet, explored audio source localisation by mobile robot, where I was able to apply my learning from MSc courses on signal processing, pattern recognition and machine learning. A novel approach based on grid occupancy was proposed to localize audio source. The results of this work was communicated to and  published at IEEE ICASSP 2015 Conference. These experiences have enhanced my research skills.   

		
		While contributing to the development of JSnoori, JTrans and J\-Safran software at LORIA\--INRIA, as mentioned in my CV, I gained expertise in different programming languages like Java, Python and JavaScript. Those projects helped me to familiarise  cluster computing environment	and tools for collaborative coding within team projects. Recently, I was introduced to TensorFlow and DL4J frameworks for implementation of deep learning architectures while investigating combination of pitch estimation with  neural networks. These demonstrate my ability to easily adapt to new R\&D environments.    
		
		The above experiences have made me well aware of  research challenges. As an individual, I also got the opportunity to develop my interpersonal communication and team 		
skills. Working on multi-disciplinary projects involved coordination within  team as well as collaboration with other teams. During my stay at LORIA\--INRIA I taught a web programming course at IUT Charlemagne. This was a great experience and helped me become a better communicator. 

	Being in the domain of research and working  with researchers for over two years gave me the urge to pursue a research career. The challenges and the perpetual evolution associated  with research only increases my desire to continue as a researcher. I believe that the PhD position will contribute enormously to my future research plan. 
	
	I am confident that my fluency in multi languages such as French, English, Berber, Arabic and my basic understanding of German language  will also enable me to contribute well to the project.
 
	Thank you for your time. Looking forward to discuss with you in this regard.
	
Sincerely,

Aghilas SINI
	
		
\end{document}