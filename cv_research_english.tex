%%%%%%%%%%%%%%%%%%%%%%%%%%%%%
% Auteur : Aghilas Sini %
% Cr�� le : 18/06/2013    %
% Version : 1.8             %
%%%%%%%%%%%%%%%%%%%%%%%%%%%%%%%%%
% Commandes pour la compilation	%
% latex cv.tex                 	%
% dvips -Ppdf -t a4 cv.dvi     	%
% ps2pdf cv.ps                 	%
%%%%%%%%%%%%%%%%%%%%%%%%%%%%%%%%%

\documentclass[a4paper,oneside]{resume}


\usepackage[english]{babel} 
\usepackage[latin1]{inputenc}  % les accents dans le fichier.tex
\usepackage[dvipsnames]{xcolor}
\usepackage{vmargin} %left, top, right, bottom
\usepackage{graphicx} % Pour ins�rer des images
\usepackage{wrapfig} % Pour placer les images
\usepackage{hyperref}
\setmarginsrb{1.2cm}{1.0cm}{1.0cm}{1.2cm}{1.2cm}{0.cm}{0cm}{0cm}

\usepackage{relsize}

%%%%%%%%%%%%%%%%%%%%%%%%%%%%%%%%%%%%
% D�finition de quelques macros    %
%%%%%%%%%%%%%%%%%%%%%%%%%%%%%%%%%%%%

% ligne horizontale sur toute la page. Usage : \ligne{Largeur}
\newcommand{\ligne}[1]{\rule[0.5ex]{\textwidth}{#1}\\}
\newcommand{\interRubrique}{\bigskip}
\newcommand{\styleRub}[1]{\noindent\textbf{\large #1}\par}
\newcommand{\indentStd}{\noindent\hspace{\lenA}}

\colorlet{Mycolor1}{blue!70!green!90!}
%%%%%%%%%%%%%%%%%%%%%%%%%%%%%%%%%%%
% Commandes Personnalis�es    	  %
%%%%%%%%%%%%%%%%%%%%%%%%%%%%%%%%%%%

% Personnalisation du titre avec un cadre gris�.
\newcommand{\mytitle}[1]{
    \begin{center}
    {\large \parashade[.9]{sharpcorners}{\textbf{#1 \vphantom{p\^{E}}}}}
    \end{center}
}



%%%%%%%%%%%%%%%%%%%%%%%%%%%%%%%%%%%%%
% L'environnement "rubrique" 
%
% Usage : \begin{rubrique}[Ind3.4entation]{Titre} [...] \end{rubrique}
% Ensuite, la premi�re colonne contient par exemple les dates, la seconde
% le descriptif.
% Par exemple :
%
% \begin{rubrique}{3.5cm}{pipotage}
% 1999--2000 	& ligne 1\\
% 		& ligne 2\\
% 1998--1999	& ligne 1\\
% [etc...]
% \end{rubrique}
%%%%%%%%%%%%%%%%%%%%%%%%%%%%%%%%%%%%%

\newenvironment{rubrique}[2][\linewidth] {
    \styleRub{#2}
    \setlength{\lenB}{#1}
    \setlength{\lenC}{\linewidth}
    \addtolength{\lenC}{-\lenA}
    \addtolength{\lenC}{-\lenB}
    \addtolength{\lenC}{-\parindent}
    \addtolength{\lenC}{-9pt}
    \indentStd\begin{tabular}[t]{p{\lenB}p{\lenC}}
}
{\end{tabular}}

\newenvironment{header}[2][\linewidth] {
    \styleRub{#2}
    \setlength{\lenB}{#1}
    \setlength{\lenC}{\linewidth}
    \addtolength{\lenC}{-\lenA}
    \addtolength{\lenC}{-\lenB}
    \addtolength{\lenC}{-\parindent}
    \addtolength{\lenC}{-9pt}
    \indentStd\begin{tabular}[t]{p{\lenB}ll}
}
{\end{tabular}}



%%%%%%%%%%%%%%%%%%%%%%%%%%%%%%%%%%%%%%%%%%%%
% Commandes utilisables dans le descriptif %
%					   %
% Modifiables � loisir... 		   %
%%%%%%%%%%%%%%%%%%%%%%%%%%%%%%%%%%%%%%%%%%%%

\newcommand{\lieu}[1]{\small{\textsl{#1}\ }}
\newcommand{\activite}[1]{\textbf{#1}\ }
\newcommand{\comment}[1]{\textsl{#1}\ }
\newcommand*\bull{\ \ \raisebox{-0.365em}[-0.15em][-0.15em]{\textscale{4}{$\cdot$}} \ } % Custom bullet point for separating content


%%%%%%%%%%%%%%%%%%%%%%%%%%%%%%%%%%%%%%%%%%
% D�but du CV proprement dit (ouf ! :) ) %
%%%%%%%%%%%%%%%%%%%%%%%%%%%%%%%%%%%%%%%%%%
\name{Aghilas SINI} % Your name
\address{615 Rue du Jardin Botanique \\ Villers les Nancy , 54600 France} % Your Your secondary addess (optional)
\address{(+33) 07~$\cdot$~77~$\cdot$~33~$\cdot$~37~$\cdot$~64 \\ sini.aghilas@gmail.com} % Your phone number and email
%\address{(+33) 03~$\cdot$~83~$\cdot$~59~$\cdot$~20~$\cdot$~97 \\ aghilas.sini@inria.fr} % Your phone number and email

\pagestyle{empty} % pour ne pas indiquer de num�ro de page...
\begin{document}
\newlength{\lenA} % indentation au d�but d'une ligne
\setlength{\lenA}{0.cm}
\newlength{\lenB} % Taille champ dates
\newlength{\lenC} % Taille champ description



%%%%%%%%%%%%%%%%%%%%%%%%%%%%%%%%%
% en-t�te 			%
%%%%%%%%%%%%%%%%%%%%%%%%%%%%%%%%%	

   
 

\begin{rubrique}[21cm]{\textcolor{Mycolor1}{\underline{Research Interest}}}\\
	Speech and Language Processing, Machine learning.
\end{rubrique}        
\interRubrique

%\begin{rubrique}[3.4cm]\underline{Education}}

%\end{rubrique} 
% \interRubrique
 

\begin{rubrique}[3.4cm]{\textcolor{Mycolor1}{\underline{Education}}}
\end{rubrique}  \\
 \begin{rubrique}[2cm]{\textit{ Academic background}}
 \\
    2014
    & \activite{Master 2 \-- Artificial Intelligence, Pattern Recognition and Robotics}\\
    & \lieu{Universit� Paul SABATIER- TOULOUSE.} \comment{France} \\ \\
   
    2013
    & \activite{Master 1 \-- Real Time Systems Engineering} \\
     & \lieu{Universit� Paul SABATIER- TOULOUSE.}\comment{France}  \\ \\
  
    2011 		
     & \activite{B.Sc \--	Control System and Automation}\\
    & \lieu{Universit� Mouloud MAMMERI- TIZI OUZOU.} \comment{Algeria}
\end{rubrique}
\interRubrique
	\begin{rubrique}[2cm]{\textit{Workshop and Summer School attended }}\\
   Nov &\activite{Workshop on ``Feedback in Pronunciation Training``}\\   
   2015 &\lieu{Hofgut Imsbach, Northern Saarland, Germany}
 \\ \\
   Jul &\activite{Speech Synthesis :Advancements in Modern Speech Synthesis Engines}\\   
   2016 &\lieu{University of Crete, Heraklion, Crete, Greece}
    \end{rubrique}  
 \interRubrique
 \\





\begin{rubrique}[3.4cm]{\textcolor{Mycolor1}{\underline{Professional Experience}}}
\end{rubrique}
\\ 
\begin{rubrique}[2.0cm]{\textit{As a research engineer}}
\\
 Jan 2016 present &\activite{IFCASL Project (Individualized Feedback for Computer-Assisted Spoken Language 
  Learning)} \\ 
	 &\lieu{LORIA\--INRIA Laboratory Nancy.}\\
		&\--  Automatic pronunciation diagnosis and error detection for Germans learning french\\	
		 &\-- Modification and re-synthesis  of learner audio samples using teacher audio samples based on Pitch Synchronous Overlap and Add algorithm\\
		&\-- Feedback to correct devoicing of final consonants in French  spoken by German learners\\
		&\-- Improving speech text alignment for language learning using deep neural network, training network with TensorFlow and decoding with DL4J (IFCASL Corpus)\\ 		
		&\-- Extraction of speech features  for pitch detection using deep neural network\\ 
\\
Jan 2016 &\activite{Lecturer in Web Programming}\\
Mar 2016	 &\lieu{IUT Charlemagne, Universit� de Lorraine, Nancy}\\
		 &\-- JavaScript, Ajax, JQuery. For second year post
		  baccalaureat students\\ \\

Nov 2014 
 &\activite{ORTOLANG Project Open Resources and TOols for LANGuage}\\ 			Jan 2016&\lieu{LORIA\--INRIA Laboratory Nancy.}\\
	&\-- Development of syntactic-semantic analyser tool called J\--Safran for spoken documents in  French language\\
	&\-- Contributing to development of  JTRANS tool for semi\--automatic alignment of speech and textual corpus \\
	&\-- Contributing to development of JSnoori an Interactive tool for speech signal processing and phonetics
\end{rubrique}
   \interRubrique
   \newpage
   \begin{rubrique}[2.0cm]{\textit{ Internship} (Master Thesis)}\\
 Mar 2014 
   & \activite{Mapping of a sound environment for a mobile robot }\\
Aug 2014 	& \lieu{LORIA\--INRIA Laboratory Nancy.}\\
	& Control of a mobile robot movements to localize a sound source as quickly as possible. The belief about the source position is represented by a discrete grid and a dynamic programming algorithm  was introduced to find the optimal robot motion minimizing the entropy of the grid .\\
	\end{rubrique} 
\interRubrique
\begin{rubrique}[16cm]{\textcolor{Mycolor1}{\underline{Publications}}}
\\
(1) E. Vincent, A. Sini and F. Charpillet, ``\label{robotic}Audio source localization by optimal control of a mobile robot,`` Acoustics, Speech and Signal Processing (ICASSP), 2015 IEEE International Conference on, South Brisbane, QLD, 2015, pp. 5630-5634.\\
\\
(2) S. Ghosh, A. Sini, Y. Laprie and C. Fauth, 2016. L1-L2 interference: the case of final devoicing of French voiced fricatives in final position by German learners. To appear in Proceedings of Interspeech 2016, San Francisco. \\
\end{rubrique}
 \interRubrique
 
\begin{rubrique}[3.4cm]{\textcolor{Mycolor1}{\underline{Activities \& Independent Courses}}}\\   
\end{rubrique} 
\\
\begin{rubrique}[3.4cm]{\textit{ Other Activities}}\\
	  & \-- Data collection for IFCASL Project \\
   	 &\-- Member of Deep Learning Discussion Group at LORIA\--INRIA\\
  	 & \-- Volunteer for MRI Data Collections\\
	 & \-- Hosted the booth for demonstration of tools in ``Village des Sciences`` at LORIA 
\end{rubrique} 
\interRubrique
 \begin{rubrique}[3.4cm]{\textit{ Online Certificate Courses}}\\
   Oct 2015 
    & \activite{Digital Signal Processing }\\
   Dec 2015& \lieu{�cole polytechnique f�d�rale de Lausanne} \comment{\textbf{taught by Prof. Paolo Prandoni \& Martin Vetterli}} \\ \\
    
    %%%%%%
     Jan 2016 
    & \activite{Data Science Specialisation }\\
  May 2016  & \lieu{Johns-Hopkins University} \comment{\textbf{by Jeff Leek, Roger D. Peng \& Brian Caffo}}\\ \\
 \end{rubrique}
	\interRubrique
	\begin{rubrique}[3.4cm]{\textcolor{Mycolor1}{\underline{Computer Skills}}}\\
     \textbf{Programming} & C, C++, Java, Shell script, Jython, Python\\ 
   \textbf{Scientific tools} &  	Matlab/Octave, TensorFlow, R\\ 
   \textbf{Web}&HTML, CSS, JavaScript, Ajax, JQuery, REST\\
   \textbf{Data structure} & XML, JSON\\ 

\end{rubrique}
\interRubrique
\\


\begin{rubrique}[3.4cm]{\textcolor{Mycolor1}{\underline{Communication Skills}}}
\\
	\textbf{French} & Fluent \\ 
    \textbf{English}& Fluent \\
    
\end{rubrique}
\interRubrique
\\

\begin{rubrique}[3.4cm]{\textcolor{Mycolor1}{\underline{References}}}
\\
\textbf{Denis Jouvet}  & Research Director, INRIA Multispeech\\
					& denis.jouvet@inria.fr \\
						 
\textbf{Yves Laprie} & Research Director, CNRS Multispeech\\      
						& yves.laprie@loria.fr \\

\end{rubrique} 
\end{document}

