\documentclass[11pt]{beamer}
\usetheme{Berkeley}
\usepackage[utf8]{inputenc}
\usepackage[english]{babel}
\usepackage{amsmath}
\usepackage{amsfonts}
\usepackage{amssymb}
\usepackage{graphicx}
\author{Aghilas}
\title{Experiments}
%\setbeamercovered{transparent} 
%\setbeamertemplate{navigation symbols}{} 
%\logo{} 
%\institute{} 
%\date{} 
%\subject{} 
\begin{document}

\begin{frame}
\titlepage
\end{frame}
\begin{frame}{Tool Experiment}
 
\end{frame}
\begin{frame}{Instruction}
\section{instruction}
\subsection{Subject Instruction}
\subsection{}

\end{frame}

\begin{frame}{Experimental Protocol }
\begin{itemize}
	\item[•]  \textit{where?} \textit {when?}
	\item[•] Mic Position: One finguer from the lips and no other part taching 
	\item[•] date of recording.z
\end{itemize}
\end{frame}
%\begin{frame}
%\tableofcontents
%\end{frame}

\begin{frame}{Isolated Words}
%\begin{tabular}{|c|c|}
%%\hline 
%% & record $(1)$  & record $(2)$  & record $(1)$ \\
%%\hline 
%%bagage & X & X & X\\ 
\begin{tabular}{|c|c|c|c|}
\hline 
 &  $1^{st}$ record  & $2^{nd}$ record & $3^{rd}$ record \\ 
\hline 
 bagage& X & X & X \\ 
\hline 
bague
 & X & X & X \\ 
\hline 
balade & X & X & X \\ 
\hline 
base & X & X & X \\ 
\hline 
blague & X & X & X \\ 
\hline 
brave & X & X & X \\ 
\hline 
case & X & X & X \\ 
\hline 
cave & X & X & X \\ 
\hline 
crabe & X & X & X \\ 
\hline 
salade & X & X & X \\ 
\hline 
village & X & X & X \\ 
\hline 
\end{tabular} 
\end{frame}
\begin{frame}{Contrastive Words}

\end{frame}
\begin{frame}{Paragraph for fixing the mic's gain}
Paris est la capitale de la France. L’agglomération de Paris compte plus de 10 millions d’habitants. Un fleuve traverse la capitale française, c’est la Seine. Dans Paris, il y a deux îles \:  l’île de la Cité et l’île Saint-Louis.\\
Paris est la capitale économique, la capitale politique et la capitale culturelle de la France. La ville compte beaucoup de lieux célèbres dans le monde entier comme « la tour Eiffel » , « l’Arc de Triomphe » et « Notre-Dame de Paris ». Les musées parisiens aussi sont très connus. Il y a, par exemple, le musée du Louvre. C’est le plus grand musée de France. On peut voir dans le musée du Louvre des tableaux magnifiques. Le plus célèbre est certainement « La Joconde » de Léonard de Vinci.\\
Paris est une ville très touristique. Chaque année, des millions de touristes du monde entier marchent sur les Champs-Élysées. Ils séjournent à l’hôtel, louent des chambres d’hôtes ou des appartements pour une semaine.




\end{frame}

\end{document}